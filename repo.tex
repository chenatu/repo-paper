\documentclass{acm_proc_article-sp}

\begin{document}

\title{NDN Repo: An NDN Persistent Storage Model}

\numberofauthors{3}

\author{
% 1st. author
\alignauthor
Shuo Chen\\
  \affaddr{Research Institue of Information}\\
  \affaddr{Tsinghua University}\\
  \affaddr{Beijing, China}\\
  \email{chenatu2006@gmail.com}
}
\maketitle

\begin{abstract}
NDN repository (Repo for short) is persistent storage model of Named Data Networking NDN compared with NDN Content Store. NDN makes in-netork storage possible because of naming and signature of network layer data packet. NDN repo is not a built-in part of NDN, but built upon application layer without tweaking NDN protocol. A specification of NDN Repo is designed to standarize repo operation interfaces. Fetching, insertion and deletion of data objects are supported in this specification and basic semantics of repo commands is defined.

In this paper, details of NDN repo specification are demonstrated and an initial implementation based on ndn-cpp library and sqlite3 is presented. (Something about the evaluation results)
\end{abstract}

\section{Introduction}


\end{document}